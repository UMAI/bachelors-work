\documentclass[12pt]{report}

%%%%% PACKAGES %%%%%
\usepackage{polski}
\usepackage[utf8]{inputenc}
\usepackage{cite}
\usepackage{titlesec}
\usepackage{url}
\usepackage{hyperref}
\usepackage{listings}
\usepackage{mathtools}
\usepackage{xcolor}
\usepackage[
	separate-uncertainty = true,
	multi-part-units = repeat
]{siunitx}
\usepackage{amsthm}
\usepackage{amsmath}
\usepackage{amssymb}
\usepackage{graphicx}
\usepackage[font=scriptsize]{caption}
\usepackage{subcaption}
\usepackage{float}
\usepackage{etoolbox}
\usepackage{indentfirst}

%%%%% CONFIGURATION %%%%%
\hypersetup{
    colorlinks,
    linkcolor={red!50!black},
    citecolor={blue!50!black},
    urlcolor={blue!80!black}
}
\titleformat{\chapter}{\normalfont\huge}{\thechapter.}{20pt}{\huge}
\setlength\parindent{2em}
\lstdefinelanguage{JavaScript}{
	keywords={typeof, new, true, false, catch, function, return, null, catch, switch, var, if, in, while, do, else, case, break},
	keywordstyle=\color{blue}\bfseries,
	ndkeywords={class, export, boolean, throw, implements, import, this},
	ndkeywordstyle=\color{darkgray}\bfseries,
	identifierstyle=\color{black},
	sensitive=false,
	comment=[l]{//},
	morecomment=[s]{/*}{*/},
	commentstyle=\color{purple}\ttfamily,
	stringstyle=\color{red}\ttfamily,
	morestring=[b]',
	morestring=[b]"
}
\lstset{
	language=JavaScript,
	backgroundcolor=\color{white},
	extendedchars=true,
	basicstyle=\footnotesize\ttfamily,
	showstringspaces=false,
	showspaces=false,
	numbers=left,
	numberstyle=\footnotesize,
	numbersep=9pt,
	tabsize=2,
	breaklines=true,
	showtabs=false,
	captionpos=b
}
\newtheorem{theorem}{Twierdzenie}
\newtheorem{definition}{Definicja}
\AfterEndEnvironment{theorem}{\noindent\ignorespaces}
\AfterEndEnvironment{definition}{\noindent\ignorespaces}
\newcommand{\R}{\mathbb{R}}

%%%%% DOCUMENT START %%%%%
\begin{document}

\begin{titlepage}
	\begin{center}
	\large 
	Uniwersytet Jagielloński\\
	Wydział Matematyki i Informatyki\\
	kierunek: matematyka komputerowa
	
	\vspace{5.0cm}
	\huge
	Układ Lorenza
	
	\vspace{1.0cm}
	\large
	Oleksandr Malakhov \\
	nr indeksu: 1130447
	\end{center}
	
	\vspace{3.0cm}
	\begin{flushright}
	promotor: dr Tomasz Kapela
	\end{flushright}
	
	\vspace{3.5cm}
	\begin{center}
		\footnotesize
		Kraków 2018
	\end{center}
	\tiny\textit{Opracowano zgodnie z Ustawą o prawie autorskim i prawach pokrewnych z dnia 4 lutego 1994 r. (Dz.U. 1994 nr 24 poz. 83) wraz z nowelizacją z dnia 25 lipca 2003 r. (Dz.U. 2003 nr 166 poz. 1610) oraz z dnia 1 kwietnia 2004 r. (Dz.U. 2004 nr 91 poz. 869)}
\end{titlepage}

\tableofcontents

\chapter{Wstęp}
	\par Celem tej pracy jest zaprezentowanie podstawowych faktów o układzie Lorenza i jego właściwościach. Jako część pracy stworzone zostało oprogramowanie umożliwiające wizualizację atraktora Lorenza oraz demonstracja wspomnianych powyżej własności.

	\section{Historia}
	\par Układ Lorenza, bez wątpliwości najlepiej znany z wszystkich chaotycznych równań różniczkowych, został sformułowany w 1963 roku przez Edwarda Lorenza jako bardzo uproszczony model konwekcji atmosferycznej. Lorenz rozważał dwuwymiarową płynną cząstkę, która była podgrzewana od dołu i chłodzona od góry. Ruch płynu może być opisany układem równań różniczkowych zawierającym nieskończenie wiele zmiennych. Lorenz zrobił ogromne uproszczające założenie że wszystkie zmienne oprócz trzech będą stałe. Te zmienne mierzą konwekcję (\textit{x}), oraz poziomową i pionową wariację temperatury (\textit{y} i \textit{z}). Wynikowy ruch prowadził do trójwymiarowego układu równań różniczkowych z trzema parametrami: liczbą Prandtl'a $\sigma$, liczbą Rayleigh'a \textit{r} oraz parametrem \textit{b}, odnoszącym się do fizycznego rozmiaru systemu:
		\[ x' = \sigma (y - x) \]
		\[ y' = rx - y - xz \]
		\[ z' = xy - bz \]

	\par Układ ten posiada to co stało się znane jako "dziwny atraktor". Przed tym jak model Lorenza zyskał sławę, jedynymi znanymi typami atraktorów stabilnych były punkty równowagi i orbity zamknięte. Układ Lorenza otworzył nowe horyzonty we wszystkich dziedzinach nauki i inżynierii, ponieważ większość zjawisk obecnych w układzie Lorenza później zostały odnalezione w biologii, teorii obwodów, mechanice itd. \cite{HSD}
	
	\section{Podstawowe definicje}
	\begin{definition}
		Bifurkacja - zjawisko skokowej zmiany własności modelu matematycznego przy drobnej zmianie jego parametrów.
	\end{definition}
	\begin{definition}
		Przestrzenią fazową nazywamy przestrzeń, w której zbiór wszystkich stanów układu przedstawiono tak, że każdemu możliwemu stanowi układu odpowiada jeden i tylko jeden punkt tej przestrzeni.
	\end{definition}
	\begin{definition}
		Układem dynamicznym nazywamy zbiór elementów, dla którego zadano funkcjonalną zależność pomiędzy czasem a pozycją w przestrzeni fazowej każdego elementu układu.
	\end{definition}
	\begin{definition}
		Niech $\phi_t (X_0)$ oznacza rozwiązanie spełniające zagadnienie początkowe $X_0$, tzn. $\phi_0 (X_0) = X_0$. Funkcję $\phi(t, X_0) = \phi_t(X_0)$ nazywamy potokiem równania różniczkowego.
	\end{definition}
	\begin{definition}
		Sekcją Poincarégo nazywamy sekcję potoku $\phi$ w przestrzeni fazowej transwersalną do pola wektorowego układu dynamicznego.
	\end{definition}
	\begin{definition}
		Niech mamy sekcję Poincarégo $\Sigma_1, \Sigma_2$ potoku $\phi_t$. Odwzorowaniem Poincarégo nazywamy odwzorowanie $P : \Sigma_1 \to \Sigma_2$ takie, że dla $x \in \Sigma_1$ i $y \in \Sigma_2$, gdzie $y$ jest pierwszym przecięciem trajektorii wychodzącej z $x$ z $\Sigma_2$, mamy $P(x) = y$.
	\end{definition}
	\begin{definition}
		Niech $X' = F(X)$ będzie układem równań różniczkowych w $\R^n$ z potokiem $\phi_t$. Zbiór $\Lambda$ nazywa się atraktorem jeśli:
		\begin{itemize}
			\item $\Lambda$ zwarty i niezmienniczy
			\item istnieje otwarty zbiór $U$ zawierający $\Lambda$ taki że dla każdego $X \in U$, $\phi_t(X) \in U$ dla wszystkich $t \ge 0$ oraz $\cap_{t \ge 0} \phi_t(U) = \Lambda$
			\item (tranzytywność) dla danych punktów $Y_1, Y_2 \in \Lambda$ i dowolnych otoczeń $U_j$ około $Y_j$ w $U$, istnieje krzywa rozwiązania, która zaczyna się w $U_1$ i później przechodzi przez $U_2$
		\end{itemize}
	\end{definition}
	\begin{definition}
		Niech $(X, \rho_x), (Y, \rho_y)$ będą przestrzeniami metrycznymi. Odwzorowanie $f: X \to Y$ spełnia warunek Lipschitza ze stałą L, gdy dla dowolnych $x_1, x_2 \in X$ zachodzi nierówność
		\[ \rho_y(f(x_1), f(x_2)) \le L \rho_x(x_1, x_2) \]
	\end{definition}

\chapter{Układ Lorenza}
	\par Jak zostało napisane wcześniej układem Lorenza nazywamy układ trzech równań różniczkowych:
		\[ x' = \sigma (y - x) \]
		\[ y' = rx - y - xz \]
		\[ z' = xy - bz \]
	Będziemy oznaczać ten układ jako $X' = \mathcal{L}(X)$. Zakładamy o parametrach że są dodatnie oraz $\sigma > b + 1$. Klasyczne wartości tych parametrów, które doprowadziły do odkrycia Lorenza to $\sigma = 10, b = 8/3, r = 28$ \cite{Lorenz}. Przez $X$ będziemy oznaczać trójkę $(x, y, z)$.

	\section{Podstawowe fakty}
	\par Układ Lorenza ma trzy punkty równowagi: początek układu współrzędnych oraz
		\[ Q_{\pm} = (\pm \sqrt{b(r - 1)}, \pm \sqrt{b(r - 1)}, r - 1) \]
	Ostatnie dwa istnieją oczywiście tylko gdy $r > 1$. Przy $r < 1$ punktem równowagi jest tylko początek, a przy $r = 1$ mamy bifurkację pitchfork. W danym przypadku mamy przejście pomiędzy jednym punktem równowagi a trzema.
	
	\par Po linearyzacji otrzymujemy układ
		\[ Y' = \begin{bmatrix}
					-\sigma & \sigma & 0 \\
					r-z & -1 & -x \\
					y & x & -b
				\end{bmatrix}
		\]
	Wartości własne dla tej macierzy dla punktu $(0, 0, 0)$ wynoszą
	\begin{center}
		$-b$ oraz $\lambda_{\pm} = \frac{1}{2} (-(\sigma + 1) \pm \sqrt{(\sigma + 1)^2 - 4\sigma (1 - r)})$
	\end{center}
	Zauważmy że obie wartości $\lambda_{\pm}$ są ujemne kiedy $0 \le r < 1$, zatem początek układu współrzędnych jest zlewem w tym przypadku.
	\begin{theorem}
		Punkty równowagi $Q_{\pm}$ są zlewami przy
		\[ 1 < r < r* = \sigma (\frac{\sigma + b + 3}{\sigma - b - 1}) \]
	\end{theorem}
	
	\par Pole wektorowe Lorenza $\mathcal{L}(X)$ jest symetryczne. Dla symetrii $S(x, y, z) = (-x, -y, z)$ mamy $S(\mathcal{L}(X)) = \mathcal{L}(S(X))$. To znaczy że odbicie przez oś $z$ zachowuje pole wektorowe. W szczególności jeśli $(x(t), y(t), z(t))$ jest rozwiązaniem układu Lorenza, to $(-x(t), -y(t), z(t))$ również jest jego rozwiązaniem.
	\\
	\par Niech $V(x, y, z) = rx^2 + \sigma y^2 + \sigma (z - 2r)^2$. Zauważmy że $V(x, y, z) = v > 0$ definiuje elipsoid w $\R^3$ ze środkiem w $(0, 0, 2r)$.
	\begin{theorem}
		Istnieje $v*$ takie że każde rozwiązanie, które zaczyna się poza elipsoidom $V(x, y, z) = v*$ w końcu wchodzi w ten elipsoid i zostaje wewnątrz niego.
	\end{theorem}
	%\begin{proof}
	%	Obliczamy
	%		\[ \dot{V} = -2 \sigma (rx^2 + y^2 + b(z^2 - 2rz)) = -2 \sigma (rx^2 + y^2 + b(z - r)^2 - br^2) \]
	%	Równanie
	%		\[ rx^2 + y^2 + b(z - r)^2 = \mu \]
	%	również definiuje elipsoid gdy $\mu > 0$. Jeśli $\mu > br^2$ mamy $\dot{V} < 0$. Zatem możemy wybrać wystarczająco duże $v*$ takie że elipsoid $V = v*$ ściśle zawiera elipsoid
	%		\[ rx^2 + y^2 + b(z - r)^2 = br^2 \]
	%	Wtedy $\dot{V} < 0$ dla wszystkich $v \ge v*$.
	%\end{proof}

	\section{Model dla atraktora Lorenza}
	\par Opiszemy model dla atraktora Lorenza, zaproponowany przez Guckenheimera i Williamsa \cite{GW}, który, jak zostało udowodnione przez Tuckera \cite{Tucker}, rzeczywiście odpowiada układowi Lorenza dla pewnych parametrów.
	
	\par Załóżmy że nasz model jest symetryczny względem odbicia $(x, y, z) \to (-x, -y, z)$, tak jak układ Lorenza. Najpierw umieścimy punkt równowagi w początku $\R^3$ i założymy że wewnątrz sześcianu $\mathcal{S}$ zadanego przez $\lvert x \rvert, \lvert y \rvert, \lvert z \rvert \le 5$ układ jest liniowy. Zamiast wykorzystywać wartości własne $\lambda_1$ oraz $\lambda_{\pm}$ dla prawdziwego układu Lorenza, uprościmy obliczenia zakładając wartości własne równe $-1$, $2$ i $-3$. Układ wewnątrz sześcianu jest zadany przez:
		\[ x' = -3x \]
		\[ y' = 2y \]
		\[ z' = -z \]
	\par My musimy wiedzieć jak rozwiązania przechodzą blisko $(0, 0, 0)$. Rozważmy prostokąt $\mathcal{R}_1$ na płaszczyźnie  $z = 1$ zadany przez $\lvert x \rvert \le 1, 0 < y \le \epsilon < 1$. Z płynem czasu wszystkie rozwiązania, które zaczynają w $\mathcal{R}_1$ ewentualnie dostają się do $\mathcal{R}_2$ na płaszczyźnie  $y = 1$ zadanego przez $\lvert x \rvert \le 1, 0 < z \le 1$. Zatem my mamy funkcję $h: \mathcal{R}_1 \to \mathcal{R}_2$ postaci
		\[ h \begin{bmatrix} x \\ y \end{bmatrix} = \begin{bmatrix} x_1 \\ z_1 \end{bmatrix} = \begin{bmatrix} xy^{3/2} \\ y^{1/2} \end{bmatrix} \]
	\begin{figure}[H]
		\centering
		\includegraphics[width=220px]{frombook_1.png}
		\caption{Rozwiązania przechodzą około $(0, 0, 0)$ \cite{HSD}}
		\label{fig:frombook_1}
	\end{figure}
	\par Po analogii z układem Lorenza umieszczamy dwa dodatkowe punkty równowagi na płaszczyźnie $z = 27$, jeden w $Q_- = (-10, -20, 27)$ i drugi w $Q_+ = (10, 20, 27)$. Zakładamy że proste $y = \pm 20, z = 27$ odpowiadają kierunkom stabilnym w $Q_{\pm}$ oraz że dwie inne wartości własne w tych punktach są zespolone z dodatnią częścią rzeczywistą.
	\par Niech $\Sigma$ oznacza kwadrat $\lvert x \rvert, \lvert y \rvert \le 20, z = 27$. Zakładamy że pole wektorowe wskazuje w dół wewnątrz tego kwadratu. Zatem rozwiązania spiralnie oddalają się od $Q_{\pm}$ jak w układzie Lorenza.
	\par Niech $\zeta^{\pm}$ oznacza dwie gałęzie niestabilnej rozmaitości w początku układu współrzędnych. Zakładamy że te krzywe obchodzą $\Sigma$ i wchodzą w kwadrat jak na rysunku:
	\begin{figure}[H]
		\centering
		\includegraphics[width=320px]{frombook_2.png}
		\caption{Rozwiązania $\zeta^{\pm}$ oraz ich przecięcie z $\Sigma$ w modeli dla atraktora Lorenza \cite{HSD}}
		\label{fig:frombook_2}
	\end{figure}
	\par Rozważmy teraz prostą $y = \nu$ w $\Sigma$. Jeśli $\nu = 0$, to wszystkie rozwiązania które zaczynają się w punktach na tej prostej dążą do początku układu współrzędnych z biegiem czasu. Zatem te rozwiązania nigdy nie wracają do $\Sigma$. Zakładamy że wszystkie pozostałe rozwiązania które zaczynają się w $\Sigma$ wracają do $\Sigma$. Sposób w jaki te rozwiązania wracają prowadzi nas do głównych założeń dotyczących tego modelu:
	\begin{itemize}
		\item \textit{Warunek powrotu:} Niech $\Sigma_+ = \Sigma \cap \{y > 0\}$ oraz $\Sigma_- = \Sigma \cap \{y < 0\}$. Zakładamy że rozwiązania przechodzące przez dowolny punkt $\Sigma_{\pm}$ wracają do $\Sigma$. Stąd mamy odwzorowanie Poincarégo $\Phi: \Sigma_+ \cup \Sigma_- \to \Sigma$.
		\item \textit{Kierunek zwężenia:} Dla każdego $\nu \ne 0$ zakładamy że odwzorowanie $\Phi$ przekształca odcinek $y = \nu$ z $\Sigma$ w odcinek $y = g(\nu)$ dla jakiejś funkcji $g$. Ponadto zakładamy, że $\Phi$ zwęża ten odcinek w kierunku $x$.
		\item \textit{Kierunek rozszerzenia:} Zakładamy że $\Phi$ rozciąga $\Sigma_+$ oraz $\Sigma_-$ w kierunku $y$ z współczynnikiem większym od $\sqrt{2}$ tak że $g'(y) > \sqrt{2}$.
		\item \textit{Warunek hiperboliczności:} Pomimo zwężenia i rozszerzenia zakładamy że $D\Phi$ przekształca wektory styczne do $\Sigma_{\pm}$, których nachylenie wynosi $\pm 1$, w wektory z nachyleniem $\mu > 1$.
	\end{itemize}
	\par Żeby znaleźć atraktor ograniczymy uwagę do prostokąta $R \subset \Sigma$ zadanego przez $\lvert y \rvert \le y*$. Niech $R_{\pm} = R \cap \Sigma_{\pm}$. Łatwo sprawdzić że każde rozwiązanie zaczynające się wewnątrz $\Sigma_{\pm}$ ale zewnątrz $R$ musi ewentualnie przejść przez $R$, dlatego wystarczy rozważyć zachowanie $\Phi$ na $R$.
	\begin{figure}[H]
		\centering
		\includegraphics[width=220px]{frombook_3.png}
		\caption{Odwzorowanie Poincare $\Phi$ na $R$ \cite{HSD}}
		\label{fig:frombook_3}
	\end{figure}
	\par Niech $\Phi^n$ oznacza $n$-tą iterację $\Phi$ i niech
		\[ A = \bigcap\limits^{\infty}_{n=0} \overline{\Phi^n(R)} \]
	Tutaj $\overline{U}$ oznacza domknięcie zbioru $U$. Zbiór $A$ będzie przecięciem atraktora dla potoku z $R$. Definiujemy:
		\[ \mathcal{A} = (\bigcup\limits_{t \in \R} \phi_t(A)) \cup \{(0, 0, 0)\} \]
	Dodaliśmy początek układu współrzędnych żeby $\mathcal{A}$ było zbiorem zamkniętym. Wtedy:
	\begin{theorem}[\cite{HSD}]
		$\mathcal{A}$ jest atraktorem dla modelu układu Lorenza.
	\end{theorem}
	\begin{theorem}[\cite{HSD}]
		Odwzorowanie Poincarego $\Phi$ ograniczone do atraktora $\mathcal{A}$ dla modelu Lorenza posiada następujące własności:
		\begin{itemize}
			\item $\Phi$ ma czulą zależność od warunków początkowych
			\item punkty periodyczne $\Phi$ są gęste w $\mathcal{A}$
			\item $\Phi$ jest tranzytywne na $\mathcal{A}$
		\end{itemize}
	\end{theorem}
	Odwzorowanie posiadające powyższe własności nazywamy \textit{chaotycznym}. \cite{HSD}

	\section{Chaos}
	Teraz możemy przedstawić dwa ważne twierdzenia dotyczących chaosu w układzie Lorenza.
	\begin{theorem}[\cite{MM}]
		Niech
			\[ P := \{(x, y, z) | z = 53\} \]
		Dla wszystkich wartości parametrów w dostatecznie małym otoczeniu $(s, r, b) = (45, 54, 10)$ istnieje przekrój Poincare $N \subset P$ taki że odwzorowanie Poincare $g$ indukowane przez układ Lorenza spełnia warunek Lipschitza i jest dobrze określone. Ponadto, istnieje $d \in N$ i ciągła suriekcja $\rho: Inv(N, g) \to \Sigma_2$ taka że
			\[ \rho \circ g^d = \sigma \circ \rho \]
		gdzie $\sigma : \Sigma_2 \to \Sigma_2$ jest przesunięciem dynamicznym na dwóch symbolach.
	\end{theorem}
	\begin{theorem}[\cite{MMS}]
		Rozważmy układ Lorenza z wartościami parametrów
			\[ (\sigma, r, b) \in \{(10, 28, 8/3), (10, 60, 8/3), (10, 54, 45)\} \]
		oraz płaszczyzną $P = \{(x, y, z) | z = r - 1\}$. Załóżmy
			\[ A_{(10,28,8/3)} := \begin{bmatrix}
									0 & 1 & 1 & 0 & 0 & 0 \\
									0 & 0 & 0 & 1 & 1 & 0 \\
									0 & 0 & 0 & 0 & 0 & 1 \\
									1 & 0 & 0 & 0 & 0 & 0 \\
									0 & 1 & 1 & 0 & 0 & 0 \\
									0 & 0 & 0 & 1 & 1 & 0
								\end{bmatrix}
			\]
			\[
			   A_{(10,60,8/3)} := \begin{bmatrix}
			   						0 & 0 & 1 & 0 \\
			   						0 & 0 & 0 & 1 \\
			   						1 & 1 & 0 & 0 \\
			   						0 & 0 & 1 & 0
			   					\end{bmatrix}
			   A_{(10,54,45)} := \begin{bmatrix}
			   						1 & 1 & 0 & 0 \\
			   						0 & 0 & 1 & 1 \\
			   						1 & 1 & 0 & 0 \\
			   						0 & 0 & 1 & 1
			   					\end{bmatrix}
			\]
		Wtedy dla wszystkich wartości parametrów w dostatecznie małym otoczeniu $(\sigma, r, b)$ istnieje przekrój Poincare $N \subset P$ taki że odpowiednie odwzorowanie Poincare $f$ spełnia warunek Lipschitza i jest dobrze określone. Ponadto, istnieje ciągłe odwzorowanie $\rho : Inv(N, f) \to \prod_k$, gdzie $Inv(N, f)$ jest największym zbiorem niezmienniczym w $N$ odnośnie $f$, z taką własnością że $\sum (A_{(\sigma, r, b)}) \subset \rho (Inv(N, f))$ oraz
			\[ \rho \circ f = T \circ \rho \]
		Co więcej, dla każdego periodycznego $\alpha \in \sum (A)$ istnieje takie $x \in Inv(N, f)$ które leży na trajektorii periodycznej z takim samym minimalnym periodem $\rho(x) = \alpha$.
	\end{theorem}
	\par Uproszczając można powiedzieć że oba te twierdzenia dowodzą tak naprawdę, że w atraktorze Lorenza (przy wybranych parametrach) istnieją punkty które są tak chaotyczne jak pewien ciąg symboli, dwóch w pierwszym twierdzeniu oraz 6 (dla $(10, 28, 8/3)$) lub 4 (dla $(10, 60, 8/3)$ i $(10, 54, 45)$) z odpowiednimi macierzami przejścia w drugim. Pierwsze twierdzenie jest jeszcze na tyle ważne że wniosło znaczący wkład w dziedzinę komputerowo wspieranych dowodów.

\chapter{Program dla wizualizacji atraktora Lorenza}
	\par Dla demonstracji powyżej opisanych własności układu Lorenza napisano program dla wizualizacji atraktora Lorenza.
	
	\section{Technologie}
	\par W celu uproszczenia korzystania z programu zaimplementowano go w formie strony internetowej, ponieważ przy takim podejściu  jest on wieloplatformowy i nie wymaga od użytkownika żadnych dodatkowych działań (takich jak pobieranie bibliotek, środowisk itd.), wystarczy posiadanie przeglądarki.
	
	\par Program składa się z trzech plików: \hyperref[sec:html]{index.html}, \hyperref[sec:css]{style.css} oraz \hyperref[sec:js]{lorenz-system.js}. Piersze dwa odpowiadają za wygląd strony, w trzecim znajduje się cała logika aplikacji. Dla rozwiązania układu Lorenza wykorzystano bibliotekę "Numeric Javascript" \cite{NumJS}, która dostarcza narzędzia dla rozwiązania równań różniczkowych zwyczajnych metodą numeryczną DOPRI (Dormand-Prince). Z otrzymanych wyników zostaje potem narysowany trójwymiarowy wykres za pomocą biblioteki "plotly.js" \cite{Plotly}.
	
	\section{Instrukcja}
	\par Po lewej stronie znajduję się panel konfiguracyjny, który posiada pola dla zmiany wartości parametrów $\sigma, r, b$, czasu początku i końca oraz punktu startowego rozwiązania. Poniżej znajdują się następujące przyciski:
	\begin{itemize}
		\item "NEW PLOT" - po kliknięciu rysuje wykres rozwiązania odpowiadającego obecnie wpisanym wartościom parametrów; jeśli jakieś wykresy już są narysowane, to zostają usunięte
		\item "ADD PLOT" - po kliknięciu dodaje wykres rozwiązania odpowiadającego obecnym wartościom parametrów do narysowanych wcześniej wykresów
		\item "START ANIMATION" - po kliknięciu na wszystkich narysowanych wykresach zaczyna się animacja ruchu punktu po krzywej rozwiązania; żeby punkty byli lepiej widoczne wykresy stają bardziej przezroczyste
		\item "STOP ANIMATION" - po kliknięciu zatrzymuje animację ruchu wszystkich punktów
		\item "RESTART" - po kliknięciu zaczyna animację punktów od początku 
		\item "REMOVE ALL" - po kliknięciu usuwa wszystkie wykresy
	\end{itemize}
	\begin{figure}[H]
		\centering
		\includegraphics[width=\textwidth]{interface.png}
		\caption{Wygląd aplikacji}
		\label{fig:interface}
	\end{figure}

	\section{Demonstracja możliwości programu}
	\par Dla klasycznych parametrów $\sigma = 10, r = 28, b = 8/3$ otrzymujemy dobrze znany obrazek atraktora Lorenza:
	\begin{figure}[H]
		\centering
		\includegraphics[width=250px]{attractor.png}
		\caption{Atraktor Lorenza dla klasycznych parametrów $\sigma = 10, r = 28, b = 8/3$, punktu startowego $(1, 0, 0)$ oraz odcinku czasu $[0, 100]$}
		\label{fig:attractor}
	\end{figure}
	Zobaczymy że ten "motyl" rzeczywiście jest atraktorem. Dla tego wybierzmy kilka dowolnych punktów startowych, narysujemy wykresy dla każdego z nich w jednym układzie współrzędnych za pomocą operacji "ADD PLOT", i zobaczmy że wszystkie trajektorie po jakimś czasie zbiegają się do atraktora:
	\begin{figure}[H]
		\centering
		\includegraphics[width=250px]{attractivity.png}
		\caption{Punkty startowe dla wykresów: $(1, 1, 1)$ dla niebieskiego, $(1, 30, 1)$ dla czerwonego, $(40, 30, 1)$ dla zielonego, $(40, -30, 20)$ dla pomarańczowego (wartości parametrów klasyczne, odcinek czasu $[0, 100]$)}
		\label{fig:attractivity}
	\end{figure}

		\subsection{Czuła zależność}
		W celu zaobserwowania czułej zależności od warunków początkowych można wyrysować dwa rozwiązania o bardzo zbliżonych warunkach początkowych, które już po krótkim czasie wykazują zupełnie odmienne zachowanie. Efekt taki można otrzymać na przykład dla $(x_1, y_1, z_1) = (1, 1, 1)$ oraz $(x_2, y_2, z_2) = (1, 1.0001, 1)$ i parametrów $\sigma = 16, r = 45.92, b = 4$:
		\begin{figure}[H]
			\centering
			\begin{subfigure}[b]{0.49\textwidth}
				\includegraphics[width=\linewidth]{sensitive_dependence_start.png}
				\caption{Na początek punkty są tak blisko że nakładają się na siebie}
			\end{subfigure}
			\begin{subfigure}[b]{0.49\textwidth}
				\includegraphics[width=\linewidth]{sensitive_dependence_after_time.png}
				\caption{Po jakimś czasie punkty kompletnie rozchodzą się}
			\end{subfigure}
			\caption{Punkt początkowy dla niebieskiego wykresu $(1, 1, 1)$, dla pomarańczowego $(1, 1.0001, 1)$ (wartości parametrów $\sigma = 16, r = 45.92, b = 4$, odcinek czasu $[0, 100]$)}
			\label{fig:sensitive_dependence}
		\end{figure}

		\subsection{Stabilność}
		Zobaczymy że dla $r < 24.74$ układ jest stabilny i rozwija się do jednego z dwóch punktów stałych atraktora, a dla $r > 24.74$ już przyjmuje bardziej zwykły wygląd:
		\begin{figure}[H]
			\centering
			\includegraphics[width=250px]{stability.png}
			\caption{Wykresy dla $\sigma = 10, b = 8/3$ oraz $r = 24$ (dla czerwonego) i $r = 25$ (dla niebieskiego), punktu początkowego $(1, 0, 0)$ oraz odcinku czasu $[0, 100]$}
			\label{fig:stability}
		\end{figure}

\chapter{Wnioski}
	\par W pracy przedstawiono układ Lorenza, niektóre jego własności, model upraszczający jego badanie, "dziwny atraktor" oraz dotknęliśmy związanego z nim tematu chaosu. Dla demonstracyjnych celów został napisany program do wizualizacji rozwiązań tego układu w formie strony internetowej.
	
	\par Układ Lorenza jest bardzo ważny, gdyż jest modelem dla zjawisk z różnych dziedzin, takich jak lasery, generatory, obwody elektryczne, reakcje chemiczne i inne. Mimo tego, że był badany przez wielu naukowców, nadal zostaje nie do końca zbadany a przede wszystkim jest nietrywialnym przykładem do testowania różnych algorytmów i metod numerycznych. 

\chapter{Appendix}
	Poniżej zamieszczono kod źródłowy aplikacji.
	
	\section{Plik \textit{index.html}}
	\label{sec:html}
	\lstinputlisting{./lorenz-system-visualization-js/index.html}
	
	\section{Plik \textit{style.css}}
	\label{sec:css}
	\lstinputlisting{./lorenz-system-visualization-js/style.css}
	
	\section{Plik \textit{lorenz-system.js}}
	\label{sec:js}
	\lstinputlisting{./lorenz-system-visualization-js/lorenz-system.js}
	
\bibliographystyle{alpha}
\bibliography{biblio}

\end{document}